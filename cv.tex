% (c) 2002 Matthew Boedicker <mboedick@mboedick.org> (original author) http://mboedick.org
% (c) 2003-2007 David J. Grant <davidgrant-at-gmail.com> http://www.davidgrant.ca
% (c) 2008 Nathaniel Johnston <nathaniel@nathanieljohnston.com> http://www.nathanieljohnston.com
%
% (c) 2012 Scott Clark <sc932@cornell.edu> cam.cornell.edu/~sc932
%
%This work is licensed under the Creative Commons Attribution-Noncommercial-Share Alike 2.5 License. To view a copy of this license, visit http://creativecommons.org/licenses/by-nc-sa/2.5/ or send a letter to Creative Commons, 543 Howard Street, 5th Floor, San Francisco, California, 94105, USA.

\documentclass[letterpaper,11pt]{article}
\newlength{\outerbordwidth}
\pagestyle{empty}
\raggedbottom
\raggedright
\usepackage[svgnames]{xcolor}
\usepackage{framed}
%\usepackage{tocloft}
\usepackage{amsmath,amssymb}
\usepackage{fancyhdr}



%-----------------------------------------------------------
%Edit these values as you see fit

\setlength{\outerbordwidth}{3pt}  % Width of border outside of title bars
\definecolor{shadecolor}{gray}{0.75}  % Outer background color of title bars (0 = black, 1 = white)
\definecolor{shadecolorB}{gray}{0.93}  % Inner background color of title bars


%-----------------------------------------------------------
%Margin setup

\setlength{\evensidemargin}{-0.25in}
\setlength{\headheight}{0in}
\setlength{\headsep}{0in}
\setlength{\oddsidemargin}{-0.25in}
\setlength{\paperheight}{11in}
\setlength{\paperwidth}{8.5in}
\setlength{\tabcolsep}{0in}
\setlength{\textheight}{9.5in}
\setlength{\textwidth}{7in}
\setlength{\topmargin}{-0.3in}
\setlength{\topskip}{0in}
\setlength{\voffset}{0.1in}


%-----------------------------------------------------------
%Custom commands
\newcommand{\resitem}[1]{\item #1 \vspace{-2pt}}
\newcommand{\resheading}[1]{\filbreak\vspace{8pt}
  \parbox{\textwidth}{\setlength{\FrameSep}{\outerbordwidth}
    \begin{shaded}
\setlength{\fboxsep}{0pt}\framebox[\textwidth][l]{\setlength{\fboxsep}{4pt}\fcolorbox{shadecolorB}{shadecolorB}{\textbf{\sffamily{\mbox{~}\makebox[6.762in][l]{\large #1} \vphantom{p\^{E}}}}}}
    \end{shaded}
  }\vspace{-5pt}
}
\newcommand{\ressubheading}[4]{
\begin{tabular*}{6.5in}{l@{{}\extracolsep{\fill}}r}
		\textbf{#1} & #2 \\
		\textit{#3} & {#4} \\
\end{tabular*}\vspace{-6pt}}
%-----------------------------------------------------------

\newcommand\blfootnote[1]{%
  \begingroup
  \renewcommand\thefootnote{}\footnote{#1}%
  \addtocounter{footnote}{-1}%
  \endgroup
}

\begin{document}

\begin{tabular*}{7in}{l@{\extracolsep{\fill}}r}
\textbf{\Large Andrew B. Maurer} & \texttt{andrew.b.maurer@gmail.com}  \\
Department of Mathematics & \texttt{andrewmaurer.github.io} \\
Boyd Graduate Research Studies\\
University of Georgia \\
Athens, Georgia 30602
\end{tabular*}
\\
\blfootnote{Last updated \today}


%%%%%%%%%%%%%%%%%%%%%%%%%%%%%%
\resheading{Education}
%%%%%%%%%%%%%%%%%%%%%%%%%%%%%%
\begin{itemize}

\item \ressubheading{University of Georgia}{Athens, Georgia}{Mathematics, Ph.D Candidate.}{2014 -- Present}

\begin{itemize}
	\resitem{Advisor:} Daniel K. Nakano
	\resitem{Research Area:} Cohomology of Lie superalgebras.
\end{itemize}

\item \ressubheading{University of Massachusetts}{Amherst, Massachusetts}{Mathematics, BS. Computer Science, Minor.}{2010 -- 2014}

\begin{itemize}
    \resitem{Advisor:} Farshid Hajir
    \resitem{Senior Project:} Hasse-Witt invariants of Jacobi polynomials.
\end{itemize}

\end{itemize}

%%%%%%%%%%%%%%%%%%%%%%%%%%%%%%
\resheading{Research}
%%%%%%%%%%%%%%%%%%%%%%%%%%%%%%

\begin{itemize}
    \item \textbf{Representations, cohomology, and geometry of Lie superalgebras.} 
    
    I am studying the relative cohomology ring $\operatorname{H}^\bullet(\mathfrak{g},\mathfrak{l};\mathbf{C})$ of a Lie superalgebra $\mathfrak g = \mathfrak g_{\bar 0} \oplus \mathfrak g_{\bar 1}$ relative to a reductive subalgebra $\mathfrak{l} \subseteq \mathfrak{g}_{\bar 0}$. My conjecture is that $\operatorname{H}^\bullet(\mathfrak{g},\mathfrak{l};\mathbf{C})$ is finitely generated over $\operatorname{H}^\bullet(\mathfrak{g},\mathfrak{g}_{\bar 0};\mathbf{C})$ and that there is a very nice spectral sequence abutting to this relative cohomology. Once this is established I will be able to use algebro-geometric techniques to investigate the mapping of support varieties induced by $\operatorname{H}^\bullet(\mathfrak{g},\mathfrak{g}_{\bar 0};\mathbf{C}) \to \operatorname{H}^\bullet(\mathfrak{g},\mathfrak{l};\mathbf{C})$.
    
    This research builds on work by Benson, Boe, Carlson, Friedlander, Gruson, Hochschild, Kujawa, Nakano, Parshall, and Serre.
    \item \textbf{Tropical geometry, algebra, and Grassmannians}
    
    The Grassmannian $\operatorname{Gr}(d,n)$ is often identified with the image of the Pl\"ucker embedding. This variety is isomorphic to the GIT quotient of $\operatorname{M}_{d \times n}$ by the (left) action of $\operatorname{GL}_d$. There has been much interest in defining tropical analogues of the Grassmannian, with several constructions due to Speyer and Sturmfels. With N. Giansiracusa, I have been working on an analogue that mimics the GIT construction. We have discovered many interesting similarities and many interesting differences when compared to the classical theory.
    
    This research builds on work by  Fink, G. Giansiracusa, N. Giansiracusa, Rinc\'on, Speyer, and Sturmfels.
\end{itemize}

%%%%%%%%%%%%%%%%%%%%%%%%%%%%%%
\resheading{External Talks \& Presentations}
%%%%%%%%%%%%%%%%%%%%%%%%%%%%%%

\begin{itemize}
 \item \textbf{Cohomology, Support Varieties, and Lie Superalgebras} \hfill Spring 2016 \\
   University of Colorado Boulder, Student Algebra Seminar
\end{itemize}

%%%%%%%%%%%%%%%%%%%%%%%%%%%%%%
\resheading{Internal Talks \& Presentations}
%%%%%%%%%%%%%%%%%%%%%%%%%%%%%%

\begin{itemize}
 \item \textbf{Tropical linear spaces} \hfill Fall 2015 \\
       UGA Tropical Geometry VRG
 \item \textbf{The tropical Grassmannian} \hfill Fall 2015 \\
       UGA Tropical Geometry VRG
 \item \textbf{Asymptotically good families} \hfill Spring 2015 \\
       UGA Graduate Student Seminar
 \item \textbf{Determinental complexity of the permanent} \hfill Spring 2015 \\
       UGA Student Algebraic Geometry Seminar
 \item \textbf{Construction of Grassmannian for Schubert calculus} \hfill Fall 2014 \\
       UGA Schubert Calculus on Grassmannian VRG 
 \item \textbf{Computability with an eye towards elliptic curves} \hfill Fall 2014 \\
       Elliptic Curves Discussion Section
\end{itemize}

%%%%%%%%%%%%%%%%%%%%%%%%%%%%%%
\resheading{Service}
%%%%%%%%%%%%%%%%%%%%%%%%%%%%%%
\begin{itemize}
\item \textbf{Principal Organizer} \hfill 2016 -- 2017 \\
  \textit{P.E.N.U.L.T.I.M.A.T.E. Seminar}
    \item \textbf{UGA MathCamp} \hfill Summer 2016 \\
          \textit{UGA Department of Mathematics}
    \item \textbf{Algebra Qualifying Exam Preparation Assistant} \hfill Summer 2016 \\
          \textit{UGA Department of Mathematics}
    \item \textbf{Graduate Visitation Day Organizer} \hfill Spring 2016 \\
        \textit{UGA Department of Mathematics}
    \item \textbf{President} \hfill Spring 2016 \\
          \textbf{Secretary} \hfill Fall 2015 \\
        \textit{UGA Chapter of the American Mathematical Society}
    \item \textbf{Logistic Assistance} \\
      \textit{Student Algebraic Geometry Seminar} \hfill 2014 -- 2015\\
      \textit{Summer Workshop in Algebraic Geometry} \hfill 2016 \\
      \textit{Topological Aspects of Algebra and Arithmetic Geometry} \hfill 2016
\end{itemize}

%%%%%%%%%%%%%%%%%%%%%%%%%%%%%%
\resheading{Conferences, Summer Schools, and Workshops Attended}
%%%%%%%%%%%%%%%%%%%%%%%%%%%%%%
\begin{itemize}
    \item \textbf{Topological \& Geometric Aspects of the Representation} \hfill Summer 2016 \\
          \textbf{Theory of Finite Groups (Summer School \& Workshop)}  \\
          \textit{Pacific Institute for the Mathematical Sciences}
    \item \textbf{Character Theory and the McKay Conjecture Summer School} \hfill Summer 2016 \\
          \textit{Mathematical Sciences Research Institute}
    \item \textbf{Southeastern Lie Theory Workshop} \hfill Summer 2016 \\
          \textit{University of Virginia}
    \item \textbf{Hodge Theory in Combinatorics Mini-Conference} \hfill Spring 2016 \\
          \textit{Georgia Institute of Technology}
    \item \textbf{Georgia Algebraic Geometry Symposium} \hfill Fall 2015 \\
          \textit{Georgia Intstitute of Technology}
    \item \textbf{Discrete Mathematics and Algorithms} \hfill Fall 2015 \\
          \textit{Clemson University Mini-Conference}
    \item \textbf{Georgia Algebraic Geometry Symposium} \hfill Fall 2014 \\
          \textit{University of Georgia}
    \item \textbf{Algebraic Geometry Northeastern Series} \hfill Fall 2014 \\
          \textit{University of Pennsylvania}
\end{itemize}

%%%%%%%%%%%%%%%%%%%%%%%%%%%%%%
\resheading{Teaching History}
%%%%%%%%%%%%%%%%%%%%%%%%%%%%%%
\begin{itemize}
    \item \textbf{High School MathCamp:} Graph Theory Group \hfill Summer 2016 \\
          Lecturer \hfill University of Georgia
    \item \textbf{Upward Bound:} SAT / ACT Math \hfill Summer 2014 \\
          Teacher \hfill Upward Bound Summer Program
    \item \textbf{Math 300:} Introduction to Proofs \hfill Fall 2013 -- Spring 2014 \\
          Teaching Assistant \hfill University of Massachusetts
    \item \textbf{Math 127 \& 128:} Calculus I \& II \hfill Fall 2011 -- Spring 2013 \\
          Teaching Assistant \hfill University of Massachusetts
    \item \textbf{Math 235:} Linear Algebra \hfill Fall 2011 -- Spring 2012 \\
          Supplemental Instruction Leader \hfill University of Massachusetts
\end{itemize}

%%%%%%%%%%%%%%%%%%%%%%%%%%%%%%
\resheading{Computer Skills}
%%%%%%%%%%%%%%%%%%%%%%%%%%%%%%
\begin{itemize}
    \item \textbf{General Programming:} Java and Python.
    \item \textbf{Mathematical Programming:} Sage, Pari/GP, Magma, R, and Julia.
    \item \textbf{Scripting Languages:} Perl, \texttt{bash}.
    \item \textbf{Markup:} \LaTeX, \texttt{org-mode}, and HTML.
    \item \textbf{Operating Systems:} Windows, Mac OS, GNU/Linux.
\end{itemize}


\end{document}


%%% Local Variables:
%%% mode: latex
%%% TeX-master: t
%%% End:
